\documentclass[a4paper, 12pt, oneside]{article}
\usepackage[T1]{fontenc}
\usepackage{ebgaramond}
\usepackage{csquotes}
\usepackage{booktabs}
\usepackage{url}
\usepackage{textalpha}
% Babel package:
\usepackage[english]{babel}

% With XeTeX$\$LuaTeX, load fontspec after babel to use Unicode
% fonts for Latin script and LGR for Greek:
\ifdefined\luatexversion \usepackage{fontspec}\fi
\ifdefined\XeTeXrevision \usepackage{fontspec}\fi

% ``Lipsiakos'' italic font `cbleipzig`:
\newcommand*{\lishape}{\fontencoding{LGR}\fontfamily{cmr}%
		       \fontshape{li}\selectfont}
\DeclareTextFontCommand{\textli}{\lishape}
\setlength{\emergencystretch}{15pt}
\usepackage{fancyhdr}
\usepackage{amssymb}
\usepackage{array}
\usepackage{float}
\usepackage{imakeidx}
\usepackage{microtype}
\begin{document}
\begin{titlepage} % Suppresses headers and footers on the title page
	\centering % Centre everything on the title page
	\scshape % Use small caps for all text on the title page

	%------------------------------------------------
	%	Title
	%------------------------------------------------
	
	\rule{\textwidth}{1.6pt}\vspace*{-\baselineskip}\vspace*{2pt} % Thick horizontal rule
	\rule{\textwidth}{0.4pt} % Thin horizontal rule
	
	\vspace{0.75\baselineskip} % Whitespace above the title

        {\LARGE The Fall of Meteorites \\ in Ancient and \\ Modern Times} % Title
	
	\vspace{0.75\baselineskip} % Whitespace below the title
	
	\rule{\textwidth}{0.4pt}\vspace*{-\baselineskip}\vspace{3.2pt} % Thin horizontal rule
	\rule{\textwidth}{1.6pt} % Thick horizontal rule
	
	\vspace{1\baselineskip} % Whitespace after the title block
	
	%------------------------------------------------
	%	Subtitle
	%------------------------------------------------
	
	{\large By Henry Alexander Miers} % Subtitle or further description
	
	\vspace*{1\baselineskip} % Whitespace under the subtitle
	
	%------------------------------------------------
	%	Editor(s)
	%------------------------------------------------
	
	\vspace{1\baselineskip} % Whitespace before the editors

        {\small A Lecture delivered in Magdalen College, Oxford; 19th Feb., 1898}
    %------------------------------------------------
	%	Cover photo
	%------------------------------------------------
	
	%\includegraphics[scale=1]{cover}
	
	%------------------------------------------------
	%	Publisher
	%------------------------------------------------
		
	\vspace*{\fill}% Whitespace under the publisher logo
	
	{\small July 1898, London}% Publication year
	
	{\small Science Progress} % Publisher

	\vspace{1\baselineskip} % Whitespace under the publisher logo

    Internet Archive Online Edition  % Publication year
	
	{\small Attribution NonCommercial ShareAlike 4.0 International } % Publisher
\end{titlepage}
\clearpage
\paragraph{}
In matters of scientific evidence relating to events which took place in early times nothing is more difficult than to place oneself in the position of a contemporary critic, amid the mental atmosphere of the time, and to regard the occurrence as it then appeared. One cannot help criticising it in the light of subsequent events, and early observers are, in consequence, too often condemned as credulous. In justice to our predecessors and to clear our own vision it is often profitable to review the development of some article of scientific belief, and to trace the steps by which it has been established. 

In the case of meteorites and the belief in their fall from the sky, the story is a curious one, for this belief, though well founded and ultimately justified, for centuries met with opposition or disregard, not from ignorant people, but from the leaders of scientific thought. 

The fairest, and doubtless the most interesting, way to gain a picture of the evidence available 100 years ago, of the impression which it produced upon thoughtful men, and of the reasoning by which they were ultimately converted, is to quote verbatim the vivid accounts of eye-witnesses, and the comments which they excited at the time. 

The following fragmentary notes contain nothing new, except that some dispersed references are perhaps for the first time brought together. 

By way of preface we may collect the main features of the evidence historical and contemporary as it presented itself to our ancestors towards the close of the last century. 

Ancient literature, of course, abounds with references, some certain and some dubious, to the fall of stones from the sky; the great stones that fell from heaven in the battle of Gibeon, the hailstones and coals of fire of the eighteenth Psalm, are among the earliest; a Chinese account relating to the year 211 B. C. describes the fall of a star which turned to stone as it fell; and still earlier Chinese records go back to the date B. C. 644. In the Talmud is a legend concerning the plague of hail in Egypt, that the hailstones were very large, each of them being about the size of an infant's head; and that as they touched the ground they burst into flames. Livy mentions several instances of a rain of stones, and in the earliest reference which he makes, in his first book, to the shower of stones that fell about 652 B. C. on the Alban Mount he is careful to distinguish them from hailstones, ``\emph{hand aliter quam} quum grandimem venti glomeratam in terras agunt, crebri cecidere cœlo lapides.'' 

The best established and the most famous of all in ancient times is that which fell about the time of the battle of Ægos Potami in B. C. 403, and near the scene of the battle, as related by Plutarch in his life of Lysander. 

Plutarch says that it was of great size and was held in great veneration by the people of the Chersonese who showed it in his own time. This fall is rendered doubly interesting by its association with the name of the philosopher Anaxagoras who is said to have foretold the event. On this subject Bayle in his Dictionary quotes Philostratus as attributing to Anaxagoras a great reputation for such predictions. At one time he predicted that on a certain day at noon the sun would become dark; at another he went to the Olympic Games with a cloak, knowing that it would rain, although the day was quite clear and serene; and a little while after it rained violently. 

As is well known, the fall at Ægos Potami is still further confirmed by Pliny, who asserts that the prediction of Anaxagoras was made sixty-two years before the battle. He goes on to say: ``The stone is still shown, of the size of a crowbar, and of a burnt colour. There was a comet at night at that time''; and further: ``A stone is at the present day held in reverence at the school of Abydos; it is only small in size, but it is the one whose fall to the earth was foretold by Anaxagoras. It is also reverenced at Cassandria, now called Potidœa.''

There can hardly be any doubt that, in spite of the legend about its prediction, all this refers to a real meteorite. The criticism of Plutarch himself on the subject is interesting. He suggests that ``shooting stars are really heavenly bodies which from some relaxation of the rapidity of their motion or by some irregular concussion are loosened, and fall not so much upon the habitable part of the earth as into the ocean, which is the reason that their substance is so seldom seen.'' 

Aristotle in his chapter on meteors has some remarks on this event in which he seems to regard the stone as having been blown by the wind; but Plutarch, who discusses the theory held by some in his own time, according to which the stone was really torn by a hurricane from the top of a mountain, expressly rejects this theory. 

Among these early accounts we find several accurate descriptions of all the phenomena which are now known to accompany the fall of a meteorite; the bright light, the noise of thunder or an explosion; and the stone itself is correctly described as of two kinds, either as a stony substance with a burnt black surface, or as metallic iron. 

Thus in the chapter preceding that in which he describes the Ægos Potami \emph{stone}, Pliny mentions the fall of a piece of \emph{iron} among the Lucani in the year before Crassus was killed by the Parthians, and he describes this as being ``spongiarum fere similis''; this expression at once recalls the aspect of several meteoric irons, notably that known as the Pallas iron which we shall have occasion to mention again. 

It is indeed more than probable that most of the iron used by primitive people who have not learnt the art of treating iron ores was derived from such masses of meteoric iron; and it is to be noticed that in Siberia, Mexico, Chili and Arabia lumps of such material were not only used for weapons, but were much prized on account of their reputed heavenly origin; Barrow in his voyages reports a mass of this sort found in the mountains behind the Cape of Good Hope which was used in this way. 

In this connection an interesting correspondence took place in 1870 between Sir John Herschel and the eminent Viennese mineralogist von Haidinger, relating to the epithet αὐτοχόωνον or ``self-fused `` applied to the iron quoit in the twenty-third book of the Iliad; the word is translated ``rudely cast'' by Liddell and Scott, but it has been suggested that it means ``native'' as opposed to forged iron. Still more curious are two lines mentioned by Eustathius as interpolated near the opening of the fifteenth book of the Iliad relating to two μύδροι or ``lumps'' cast by Zeus upon Troy, ὄφρα πέλοιτο καὶ ἐσσομένοισι πυθέσθαι; and Eustathius adds: ``Lumps of this kind are pointed out by the Periegetæ who call them anvils fallen from heaven.''

In addition to the more or less direct evidence of which the preceding are examples there is abundance of indirect evidence derived from the worship of stones; for this worship must, I think, have at least sometimes originated in a meteoric fall. 

Jevons in his \emph{Introduction to the Study of Religion} traces the origin of stone-worship and of the anointing of stones merely to the veneration of those which had been used as altars, and this appears to be the opinion of most authors upon the subject. But, although it is by no means probable that most or even many of the holy stones were meteorites, it is more than probable that when so remarkable an event as the fall of a stone from the sky did take place it must have provoked religious awe, and the stone itself must generally have become an object of worship. It is certainly remarkable that this origin was ascribed to several of the holy stones of antiquity. 

The Diana of the Ephesians of the Acts of the Apostles, the ``image that fell down from Jupiter'' is perhaps the best-known instance. 

The Kaaba, or black stone of Mecca, venerated by all Mohammedans, was worshipped by the Arabians in very early ages, and, although it has not been seen by any one specially qualified to judge, is now generally supposed to have been meteoric in origin. In Sale's introduction to the Koran it is stated that this stone was supposed to have fallen down from heaven before the Deluge. Again Maximus Tyrius says that he had actually seen a quadrangular stone which was worshipped by the Arabians, and in the same passage he mentions that the Paphians worshipped a statue of Venus which looked like a white pyramid. 

No doubt many of the holy stones were venerated on account of their form quite independently of their origin; the image of Venus in Cyprus is described by Tacitus as being not of human shape but conical; and he adds: ``Et ratio in obscuro''; and Pausanias says that the images of Jupiter Melichius and of Diana were, the one a pyramid, and the other a column. 

Even among the stones enumerated by Pliny which have been more or less identified with meteoric stones, the shape is one of the features according to which some at least were distinguished. The Ceraunia, or sky-stones, of his classification include as varieties stones which he refers to as Bætuli, Brontia and Notia, some of which have special shapes. All of these names frequently recur in mediæval literature. 

It is evident that in one passage Pliny uses Ceraunia for a variety of precious stone, Beryl or Sapphire perhaps; but besides these he quotes Sotacus for the existence of two kinds of Ceraunia ``which are black and red, resembling axes. Such as are black and round are holy things; cities and fleets can be captured by their means. A third sort greatly sought by the Magi are only found in places struck by lightning.'' 

The word Bætylus remains a mystery; the name was primarily given to the stone which Saturn swallowed in mistake for Jove, but seems to have been subsequently applied to all meteoric stones. Hesychius suggests the Hebrew ``Bethel'' and the stone of Jacob as its origin; and this derivation seems to be accepted in the \emph{Dictionary of the Bible}, though without any philological justification. 

About the Brontia Pliny says that if we have sufficient faith we are to believe that they get into the heads of tortoises after thunderstorms; and here, I think, there is some confusion between the shape of some Brontia and the origin of others. 

Through the midst of all this superstition, however, runs a continuous thread of reference to a celestial origin by which we are now able, in the light of subsequent experience, to trace a constantly recurring expression of the belief in meteoric falls. 

The last statement for example in Pliny's enumeration appears to refer to meteorites; but the remark about axes may indicate that stone celts or hammer-heads are denoted by his first class. The Cambridge authority, King, compares the German word \emph{Donnerkeil} for Thunderbolt; and again with the word \emph{Bætuli} the Saxon ``Beetle'' which means a mallet, and concludes that these names in general refer to stone implements. 

And here we are confronted by a curious complication in the history of the subject. Side by side with the fact that stones fell from the sky, existed the belief that the origin of a thing was indicated by its shape; consequently a celestial origin was ascribed to those stones whose shape resembled that of a missile, and both stony concretions, fossils such as echini, and stone celts were supposed to be meteorites. It is difficult now to disentangle the evidence of falls actually witnessed from that which is merely based upon the shape of the stones to which many of the mediæval accounts relate. 

At the present day both Belemnites and the marcasite nodules found in the chalk are popularly supposed to be thunderbolts on account of their shape. 

Conrad Gesner in his book \emph{De Figuris Lapidum} (1565) describes the various stones which derive their names from their real or supposed meteoric origin, the Ceraunias, the Chelonitis, the Brontias, the Bætylus, and gives figures of many. Some of these are obviously fossils, others are stone implements; his accurate description of some which he had received as thunderstones from Kentman shows that they are clearly the latter. But it is equally certain that some of his words relate to real meteoric stones. He makes in particular this interesting remark: ``The stone which fell from the sky in 1492 and is hung in the Church at Ensisheim and weighs 300 pounds (unless it has lost weight owing to the many visitors who take away fragments of it) has, I think, no particular shape''; and he mentions that he had actually received a piece of this stone. 

So much for the general evidence available about 300 years ago; the last reference brings us to a time when stones fell which are actually preserved at the present day, so that the veracity of contemporary accounts relating to them can no longer be questioned. From the sixteenth century onwards there are a number of such accounts in which we can now, reading by the light of subsequent experience, see internal evidence of their accuracy, and by which we are led to attach equal confidence to the accuracy of some of the earlier reports such as that of the Ægos Potami fall Omitting, therefore, a number of mediæval references which may be found in the Saxon chronicles, Eusebius, Cardanus, Avicenna, Scaliger and others, we may pass directly to the Ensisheim fall, the earliest one of which we possess a contemporary account relating to a stone that still exists and has been proved to be meteoric. 
\begin{center}
\emph{Fall of the Ensisheim Stone.}
\end{center}
\paragraph{}
The account is as follows:--- 

``On the 16th of November, 1492, a singular miracle took place. Between 11 and 12 in the forenoon with a loud crash of thunder and a prolonged noise heard afar off there fell in the town of Ensisheim a stone weighing 260 pounds. It was seen by a child to strike the ground in a field where it made a hole more than five feet deep. It was taken to the church as a miraculous object. The noise was heard so distinctly at Lucerne and many other places that in each of them it was thought that some houses had fallen. King Maximilian, who was then at Ensisheim, had the stone carried to the castle; after breaking off two pieces, one for the Duke of Austria and the other for himself, he forbade further damage, and ordered the stone to be suspended in the parish church.'' 

With this may be compared an account quoted by Sir Norman Lockyer from a rare tract in the British Museum, in which the obviously truthful statement of the occurrence is somewhat obscured by the fancy begotten by terror. 

The tract is entitled:--- 
\begin{center}
\emph{Looke up and see wonders: a miraculous Apparition in the Ayre, lately seen in Barkeshire at Bawlkin Greene neare Hatford.} And is as follows:--- 
\end{center}
\paragraph{}
``At Hatford some 8 m. from Oxford. Over this towne upon Wensday being the 9th of this instant Moneth of April, 1628, about 5 of the clocke in the after noone this miraculous, prodigious and fearefull handyworke of God was presented. A gentle gale of Wind then blowing from between the W. and N. W. in an instant was heard first a hideous rumbling in the Ayre, and presently after followed a strange and feare-full peal of Thunder running up and downe these parts of the countrey, but it strake with the loudest violence and more furious tearing of the Ayre about a place called the White Horse Hill. The whole order of this thunder carried a kind of majesticall state with it, for it maintayned (to the affrighted Beholder's seeming) the fashion of a fought Battaile. It began thus:--- First for an onset went off one great Cannon as it were of thunder above like a warning peece to the rest that were to follow. Then a little while after was heard a second; and so by degrees a third untill the number of 20 was discharged in very good order though in very great terror. In some little distance of time after this was audibly heard the sound of a Drum beating a Retreate. Amongst all these angry peales shot off from Heaven this begat a wonderful admiration that at the end of the report of every cracke or Cannon-thundering, a hizzing noise made way through the ayre not unlike the flying of bullets from the mouthes of Great Ordnance; and by the judgment of all the terror stricken witnesses they were Thunder bolts. For one of them was seene by many people to fall at a place called Bawlkin Greene being a mile and a half from Hatford; which Thunder bolt was by one Mistris Greene caused to be digged out of the grounde she being an eye-witnesse amongst many other of the manner of falling. The form of the stone is three-square and picked in the end: The colour outwardly blackish, somewhat like Iron; crusted over with that blacknesse about the thicknesse of a shilling. Within it is a soft, of a gray colour, mixed with some kind of miner-all shining like small peeces of glasse.''

With this may further be compared the record relating to a fall of iron at about the same date (1620) but in a very different part of the world. The following is a translation by Colonel Kirkpatrick from a contemporary Persian account of which he possessed the manuscript written by the Emperor Jehangire himself. 
\begin{center}
\emph{Fall of a Persian Meteorite.}
\end{center}
\paragraph{}
``Early on the 30th of Furverdeen, of the present year, and in the Eastern quarter of the heavens there arose in one of the villages of the Purgunnah of Jalindher, such a great and tremendous noise as had nearly, by its dreadful nature, deprived the inhabitants of the place of their senses. During this noise a luminous body was observed to fall from above on the earth, suggesting to the beholders the idea that the firmament was raining fire. In a short time the noise having subsided, and the inhabitants having recovered from their alarm, a courier was dispatched by them to Mahommed Syeed, the Aumil of the aforesaid Purgunnah, to advertise him of this event. The Aumil, instantly mounting his horse, proceeded to the spot where the luminous body had fallen. Here he perceived the earth, to the extent of ten or twelve guz in length and breadth, to be burnt to such a degree that not the least trace of verdure or a blade of grass remained; nor had the heat which had been communicated to it yet subsided entirely.''

``Mahommed Syeed hereupon directed the aforesaid space of ground to be dug up; when, the deeper it was dug, the greater was the heat of it found to be. At length a lump of iron made its appearance, the heat of which was so violent that one might have supposed it to have been taken from a furnace. After some time it became cold; when the Aumil conveyed it to his own habitation, from whence he afterwards dispatched it in a sealed bag to court.''

``Here I had this substance weighed in my presence. Its weight was 160 tolahs. I committed it to a skilful artisan, with orders to make of it a sabre, a knife, and a dagger. The workmen soon reported that the substance was \emph{not malleable, but shivered into pieces under the hammer}. Upon this, I ordered it to be mixed with other iron.''

``Conformably to my orders, three parts of the \emph{iron of lightning} were mixed with one part of common iron; and from the mixture were made two sabres, one knife and one dagger. By the addition of the common iron, the new substance acquired a fine temper; the blade fabricated from it proving as elastic as the most genuine blades of Ullmanny, and of the South, and bending, like them, without leaving any mark of the bend. I had them tried in my presence and found them cut excellently, as well indeed as the best genuine sabres. One of these blades I named \emph{Katai} or \emph{the cutter}; and the other \emph{Burk-serisht} or the \emph{lightning natured}.'' 

``A poet composed and presented to me on this occasion the following tetrastich:---

This earth has attained order and regularity through the Emperor Jehangire: In his time fell \emph{raw} iron from lightning: That iron was, by his world-subduing authority Converted into a dagger, a knife, and two sabres.'' 

With these early examples of the more modern and authentic records may be compared the two following which are quite modern: one relating to a meteoric stone that fell in Russia, and the other to an iron that fell in Mexico. The first has a special interest as the stone in which Diamond was found, and the second as the only modern meteorite which has been known to fall during a shower of shooting stars.
\begin{center}
\emph{Fall of the Novo-Urei Stone.} 
\end{center}
\paragraph{}
``At 7:18 A. M. on 22nd of September, 1886, some peasants were working in a field at Novo-Urei in Russia.''

``It was a dull morning without rain, although the sky was covered with clouds. Suddenly the air seemed filled with a bright light, followed in a few seconds by a violent report which was immediately succeeded by a second explosion. At the same moment the terrified peasants saw a fiery ball fall to the ground only a few yards from where they stood, and a second, but larger one was seen to descend into a neighbouring wood. The whole thing lasted less than a minute. The men fell in mortal terror to the ground and for some time dared not move. They thought that a frightful storm had burst over their heads, and that fiery thunderbolts were falling. At length they recovered courage and went to the place where the thunderbolt had fallen. To their amazement they found here, in a small cavity a black stone half embedded in the earth, and still hot. It felt very heavy. They searched in vain for the other stone in the wood; but the next day a similar stone was found in a neighbouring field.'' 
\begin{center}
\emph{Fall of the Mazapil Iron, 1885.}
\end{center}
\paragraph{}
``It was about nine in the evening when I went to the corral to feed the horses, when suddenly I heard a loud hissing noise exactly as though something red-hot was being plunged into cold water, and almost instantly there followed a somewhat loud thud. At once the corral was covered with a phosphorescent light, and suspended in the air were small luminous sparks as though from a rocket. I had not recovered from my surprise when I saw this luminous air disappear, and there remained on the ground only such a light as is made when a match is rubbed. A number of people from the neighbouring houses came running towards me, and they assisted me to quiet the horses which had become very much excited. We all asked each other what could be the matter, and we were afraid to walk to the corral for fear of being burned. When in a few moments we had recovered from our surprise we saw the phosphorescent light disappear, little by little, and when we had brought lights to look for the cause, we found a hole in the ground and in it a ball of fire. We retired to a distance fearing it would explode and harm us. Looking up to the sky we saw from time-to-time exhalations or stars which soon went out, but without noise. We returned after a little and found in a hole a hot stone which we could barely handle, which on the next day we saw looked like a piece of iron. All night it rained stars, but we saw none fall to the ground, as they seemed to be extinguished while still very high up.'' 

But it is not necessary to multiply instances. It is clear that in ancient times and in the Middle Ages meteoric falls were often recorded and were implicitly believed by ordinary people. Boetius de Boot in his book on Stones (1609) says: ``Si quis hanc vulgi opinionem refellere velit insipiens videatur.''

The preceding examples will serve as a sketch of the evidence which presented itself to scientific men in the last century. 

Meanwhile, however --- and this is the fact to which I wish particularly to draw attention because it makes the history of meteorites so curious as a study of scientific evidence --- the whole subject had with the growth of scientific knowledge become gradually discredited among thoughtful and well-educated people. Now that we know the fact to have been true, it is easy on the one hand to make allowances for the fancy which enters so largely into such past accounts as that of the Hatford fall, and on the other to reject among the present records which appear from time to time in the public press those which describe the fall of stones during thunderstorms, and under other improbable or impossible conditions, as well as the details imputed by terror and superstition. 

But before the fact was known to be true, the evidence was so vitiated by delusions of various sorts, and eye-witnesses were so apt to be deceived by the sudden nature of the event and the terror which it inspired, that those who were best able to criticise circumstantial evidence were the first to reject that relating to meteorites. 

I rather suspect that this was also so among the ancients, although the same critical attitude towards such events would hardly be expected from them. Aristotle barely alludes to thunderstones; there appears to be no mention of them in Herodotus; and Lucretius only asks why a bolt never falls when the sky is unclouded. 

In later times neither Locke, nor Bacon, nor Newton appears to make any reference to the matter; and Boyle only mentions meteorites as ``Stones which pass among the vulgar for thunderstones.'' 

At the end of the last century the leaders of scientific thought had criticised the evidence and rejected it \emph{in toto}. 

Their position is really very well expressed more than a century before by Torbernus Bergmann, the celebrated Professor of Chemistry at Upsala, in his treatise \emph{De Avertendo Fulmine} (1764), where he makes the following observation: ``Popularis erat veterum Teutonum Suionumque opinio lapides quosdam de coelo mitti, quos Thors-vigger (Donnerkeile, \emph{i. e.} Lapides Ceraunios s. Belemnitas) vocabant''; and then he states that three opinions concerning these Ceraunian stones are held among philosophers. 1. That the whole thing is a fable, and that the stones themselves are weapons in which the handiwork of man is clearly apparent; 2. that these stones really fell to the ground with the lightning, as is thought by the Arabians; in which case they may either have been carried into the air by the wind, or may have been generated in the air as is suggested by Cartesius; or 3. that they have been fused into a mass at the point where lightning has struck the ground; an argument adduced in favour of this view by Stahlius is that a certain man, expert in such matters, having found a little hole in the ground while he was digging predicted that there would be a ceraunian stone at the bottom; which proved to be the case. 

Bergmann himself rejects the first two hypotheses as clearly absurd; but being convinced by the recent discoveries of Franklin that the phenomenon is electrical, thinks that the last explanation is not only possible but probable. 

It is rather difficult now to realise the attitude of mind adopted by the leaders of thought at the beginning of the present century. There was no lack of evidence; plenty of witnesses asserted that they had seen the stones fall, and many of them were actually preserved. Shooting stars have of course always been familiar, just as they are at the present time, but the scientific authorities of that date after duly weighing the evidence came to the conclusion that there was no proof that these stars ever fell to the ground. They preferred to believe that those who professed to have witnessed such falls were mistaken, and that the supposed meteorites were ordinary stones struck by lightning. In fact, the witnesses generally mentioned thunder and lightning as accompanying the fall; this in itself was suspicious; and, further, the witnesses were evidently so scared that they hardly knew what they had seen. And yet one cannot help feeling that the available evidence, if acutely criticised, was sufficient to enable a scientific critic to extract the truth from the mass of legend in which it was embedded; and in fact this was actually done with signal success by a writer whose work opens the last chapter in the history of the belief in meteorites. 

The modern development of a scientific proof of the existence of sky-stones, as distinct from terrestrial material is no doubt familiar to many through Mr. Fletcher's admirable \emph{Introduction to the Study of Meteorites}. At the risk of considerable repetition I must give a brief sketch of the meteoric events of the last decade of the eighteenth and the first decade of the nineteenth century, with the object of showing how the evidence was received by the critics of that date, and how they were finally persuaded. The chapter of proof really begins in the year 1794, when the German physicist Chladni wrote a very remarkable paper, ``Über den Ursprung der von Pallas gefundenen und anderer ihr ähnlicher Eisenmassen.'' 

The traveller Pallas in 1772 saw in Siberia a great mass of iron weighing about 1500 pounds which had been discovered by a Cossack at the top of a mountain near Krasnojarsk in Siberia. It was spoken of by the Tartars as a holy thing fallen from heaven. There was nothing like it in the neighbourhood and it was too large to have been transported to the mountain top by human agency. It was a peculiar spongy-looking mass which strongly recalls Pliny's description quoted above. 

Chladni argued that this iron had evidently been fused, but not by man, electricity, or accidental fire, considering the place where it was found; there are no volcanoes anywhere in the neighbourhood; therefore it must have fallen from the sky. To the same origin he referred a huge mass found by Indians at Otumpa far away in the Argentine Desert of South America; a mass which was at first supposed to be an iron mine; and he suggested that other masses of native iron are also meteoric. 

Chladni even went so far as to suggest that these masses were bodies of the same sort as those which produce the appearance of a shooting star in their passage through the air. 

He subsequently fortified his views by an enumeration of a great number of reported falls of stone from the sky in ancient and mediæval times, of which I have quoted several above. 

Of course Chladni's theory was not accepted --- it was so improbable, and his arguments seemed to be only based upon the difficulty of accounting for the presence of this particular mass of iron in Siberia in any other way. His contemporaries regarded the essay as an ingenious but unconvincing piece of work. 
\begin{center}
\emph{Fall of the Sienna Stone.}
\end{center}
\paragraph{}
Immediately after the appearance of Chladni's paper, however, a remarkable event took place at Sienna in Tuscany on 16th June, 1794, at 7 o'clock in the evening. 

The event is thus described in the following letter from the Earl of Bristol to Sir William Hamilton which has been often quoted. 

``In the midst of a most violent thunderstorm about a dozen stones of various weights and dimensions fell at the feet of different persons, men, women and children. The stones are of a quality not found in any part of the Siennese territory; they fell about eighteen hours after the enormous eruption of Mount Vesuvius; which circumstance leaves a choice of difficulties in the solution of this extraordinary phenomenon. Either these stones have been generated in this igneous mass of clouds which produced such unusual thunder; or --- which is equally incredible --- they were thrown from Vesuvius at a distance of at least 250 miles: judge then of its parabola. The philosophers here incline to the first solution. I wish much, sir, to know your sentiments. My first objection was to the tact itself, but of this there are so many eyewitnesses it seems impossible to withstand their evidence.'' 
\begin{center}
Sir Wm. Hamilton (\emph{Phil. Trans.}, 85, p. 103), after quoting this letter says:--- 
\end{center}
\paragraph{}
``The outside of every stone that has been found, and has been ascertained to have fallen from the cloud near Sienna, is evidently freshly vitrified, and is black, having every sign of having passed through an extreme heat; when broken, the inside is of a light grey colour mixed with black spots, and some shining particles, which the learned here have decided to be pyrites, and therefore it cannot be a lava, or they would have been decomposed. Stones of the same nature, at least as far as the eye can judge of them, are frequently found on Mount Vesuvius; and when I was on the mountain lately, I searched for such stones near the new mouths, but as the soil round them has been covered with a thick bed of fine ashes, whatever was thrown up during the force of the eruption lies buried under those ashes. Should we find similar stones with the same vitrified coat on them on Mount Vesuvius, as I told Lord Bristol in my answer to his letter, the question would be decided in favour of Vesuvius; unless it could be proved that there had been, about the time of the fall of these stones in the Sanese territory, some nearer opening of the earth, attended with an emission of volcanic matter, which might very well be, as the mountain of Radicofani, within fifty miles of Sienna, is certainly volcanic. I mentioned to his lordship another idea that struck me. As we have proofs during the late eruption of a quantity of ashes of Vesuvius having been carried to a greater distance than where the stones fell in the Sanese territory, and mixing with a stormy cloud have been collected together just as hailstones are sometimes into lumps of ice, in which shape they fall, and might not the exterior vitrification of those lumps of accumulated and hardened volcanic matter have been occasioned by the action of the electric fluid on them? The celebrated Father Ambrogio Soldoni, professor of mathematics in the university of Sienna, is printing there a dissertation upon this extraordinary phenomenon, wherein, as I have been assured, he has decided that those stones were generated in the air, independently of volcanic assistance.'' 

Soldoni's account contains the following additional details: ``Two ladies being at Coyone, about twenty miles from Sienna, saw a number of stones fall with a great noise in a neighbouring meadow; one of which, being soon after taken up by a young woman, burnt her hand; another burnt a countryman's hat; and a third was said to strike off the branch of a mulberry tree, and to cause the tree to wither.''

Soldoni himself thought that ``the stones were generated in the air by a combination of mineral substances which had risen somewhere or other as exhalations from the earth, but not from Vesuvius.''

Very shortly afterwards (1796) appeared the work of Edward King, \emph{Remarks Concerning Stones said to have Fallen from the Clouds}, in which this and other falls were enumerated and discussed. In regard to the Sienna stones he recalls instances in which volcanic dust was known to fall upon ships 100 leagues from the scene of eruption, and quotes Sir William Hamilton's account of the Vesuvius eruption in which ashes appeared to be projected to a height of twenty-five or thirty miles; he suggests as an explanation of the Sienna stones that these ashes were carried beyond Sienna northwards, and were then brought back by a northerly wind, congealing from the air, which he had always regarded as ``the great consolidating fluid out of which all solid bodies are composed.'' 
\begin{center}
\emph{Fall of the Wold Cottage Stones.}
\end{center}
\paragraph{}
At the very time when King was writing, a stone was being exhibited in London which weighed fifty-six pounds and was seen to fall at Wold Cottage in Yorkshire on 13th December, 1795. 

The following is the account given by the handbill which accompanied the exhibition: ``It penetrated through twelve inches of soil and six inches of solid chalk rock, and in burying itself had thrown up an immense quantity of earth to a great distance; as it fell a number of explosions were heard about as loud as pistols. 

``In the adjacent villages the sounds heard were taken for guns at sea; but at two adjoining villages were so distinct of something passing through the air towards the habitation of Mr. Topham that five or six people came up to see if anything extraordinary had happened to his house or grounds. When the stone was extracted it was warm, smoked, and smelt very strong of sulphur. Its course, as far as could be collected from different accounts was from south-west. The day was mild and hazy; the sort of things very frequent in the Wold Hills where there are no winds or storms; but there was not any thunder or lightning the whole day. No such stone is known in the country. There was no eruption in the earth: and from its form it could not come from any building, and as the day was not tempestuous it did not seem possible that it could have been forced from any rocks, the nearest of which are those of Flamborough Head, a distance of twelve miles. The nearest volcano I believe to be Hecla in Iceland.'' 

It might be thought that an examination of the stones themselves would be sufficient to prove or to disprove the common belief about their origin; and about this time an examination of the sort was undertaken by some of the leading French chemists, who actually made an analysis of the Ensisheim stone, and, finding it to contain nothing new, concluded that it was terrestrial. Their report on these supposed sky-stones terminated with the words: ``Ignorance and superstition have attributed to them a miraculous existence at variance with the first notions of natural philosophy.'' 
\begin{center}
\emph{Fall of the Benares Stone.}
\end{center}
\paragraph{}
In the year 1798, another well-authenticated fall took place in India, fourteen miles from Benares, where a luminous meteor was observed in the western heavens at 8 P. M. accompanied by a loud noise resembling thunder. The sky was perfectly serene; not the smallest vestige of a cloud had been seen for about eight days, nor were any seen for many days after. ``Inhabitants observed that the light and thunder were accompanied by the noise of heavy bodies falling. Uncertain whether some of their deities might not have been concerned in this occurrence they did not venture out to inquire into it until the next morning, when the first circumstance which attracted their attention was the appearance of the earth being turned up in different parts of their fields, where on examining they found the stones.'' 

Again in the same year a fall was reported at Ville-franche, near Lyons; the meteor was seen by many people and the eyewitnesses were horribly alarmed. One man whose house was within twenty paces of the spot where the stone fell was so terrified by the noise that he ``shut himself up with his family in the cellar, and then in the bed-chamber, where, fear prevailing over curiosity, he spent the night without daring to go out to examine what had happened.'' 

By this time Chladni's memoir had attracted attention to at any rate the possibility of the truth of such reports, and all these recent occurrences gave rise to much discussion. It will be sufficient to quote a few of the contemporary criticisms in order to gain some idea of the prevailing impression which they created among those who read them. 

W. Beauford writing in the \emph{Philosophical Magazine} in 1802, concludes that the matter must be of volcanic origin and derived either from Vesuvius, Etna, or Hecla. But the distances are too far for them to have traversed as stones. ``Hence, if they originate from volcanic ashes they must be formed in the clouds where those ashes meeting with carbonic, sulphuric and other acids, and mixing with earthy particles drawn from terrestrial objects are by the electric fluid in the lightning precipitated from the aqueous vapours which bore them up, and, becoming united, fall to the earth in the form of stones, as in some measure is evinced from the flashes of light and detonation which accompany their fall.'' 

Pictet writing on behalf of the French National Institute in 1803 expressed the opinion that ``the attention of philosophers should be directed to the subject in order that the phenomenon if true may be confirmed --- or if only an illusion supported by popular error may be consigned for ever to the class of errors.'' In the same year the French Institute mentions new motives to ``induce philosophers to examine and appreciate the different testimonies in consequence of which the stones in question have been supposed to have fallen from the clouds. When a phenomenon is announced if we were able to ascertain by a complete enumeration of the different physical agents that none of them is capable of producing it the impossibility of the phenomenon would be the inevitable result and consequently the falsity of the account. But on the other hand, when we find a cause which establishes the possibility of it, if sound logic forbids us to ascribe it exclusively to this cause, it commands us at the same time to substitute doubt for complete negation and to employ every means possible of confirming the fact, because it is not repugnant to the general laws of Nature.'' 

This very guarded and somewhat curious statement is explained by the fact that Laplace and Poisson had calculated that a body projected from the moon would require only a velocity five times as great as that of a bullet of a twenty-four pounder, discharged with a quantity of gunpowder equal to half its own weight, to reach the earth after a journey of sixty-four hours, and would arrive with a velocity of 31,000 feet a second. It is evident that the accounts of the falls themselves were by this time no longer discredited, and that even the lightning theory was losing its adherents. 

In 1803 Olbers, who had at first asserted that the Sienna stones were from Vesuvius, is led by the similarity of the sky-stones in different parts of the world to agree that they had a common origin and probably came from the moon. The chemist Vauquelin also inclined to the moon theory; it is evident that the absence of atmosphere there would account for the stones leaving a lunar volcano without retardation and also without experiencing oxidation. Writing of the Barbotan fall which took place in 1789 he says: ``Some peasants brought stones which they said were the result of the fall of the meteor; but at that period they were laughed at. What they said was considered as fables --- and those to whom the stones were offered would not accept of them. The peasants would now have more reason to laugh at the philosophers.'' 

Even at this period, however, when it began to be suspected that stones really fell from the sky and that they may have a common origin, it was by no means universally conceded that they were extraterrestrial.

Proust, in a paper published in the \emph{Journal de Physique} in 1805 (reported in \emph{Nicholson's Journal}, vol. 12.), describes a stone which fell in 1773 at Sena in the district of Sigena, in Spain; and gives the results of an analysis. He concludes that such stones ``cannot subsist in any of the habitable parts of the globe. But from the eternal cold of the polar regions, where water remains for ever a solid mass, and iron cannot rust, we may reasonably look to these regions as the native place of such bodies.''

But we can now hurry to the close of the story.

It is pretty evident from the preceding quotations that at the beginning of the present century the attitude of scientific men towards the reported fall of meteorites was one of suspicious indifference. There might be something in it all; there was fair evidence in many cases that something startling had happened; but no reliance could be placed upon the evidence of the senses under such conditions; and the witnesses were generally ignorant rustics.

It had been proved by Franklin that lightning is the same as the electric spark; and thunder is an accompaniment of lightning. The witnesses of these events professed to have heard thunder; what they saw and found were, no doubt, ordinary stones struck by lightning; and this conclusion seemed to be supported by chemical and mineralogical study of the stones themselves.

In the meantime an English chemist was, unnoticed, pursuing the only satisfactory method of completing the scientific proof which had been initiated by Chladni's acute reasoning.

This chemist, Edward Howard by name, collected pieces of four stones, those which fell at Sienna, Wold Cottage, Benares, and one which fell during a thunderstorm in 1753 in Bohemia. He made analyses of them and submitted them for mineralogical investigation to the Count de Bournon.

The results of his long and patient investigation were communicated to the Royal Society in 1803. He concluded that all these four stones had nearly the same chemical composition; and that though there was nothing actually new in them, their mineral composition was so unlike that of all terrestrial stones, and so similar for the four masses --- though they came from widely distant places and were asserted to have fallen at very different dates --- that they must have had a common origin; and he concluded, though with diffidence, that they may very possibly be really meteoric.

This paper attracted much attention in the scientific world, and the opportunity for putting it to the test soon occurred in France, where the new views met with the greatest opposition. A shower of stones fell on 26th April, 1803, at L'Aigle in the department of Orne. The eminent physicist Biot was sent down by the French Academy to investigate the matter, and reported that there was no doubt that a violent explosion was heard that day for seventy-five miles round; that a fire ball was seen, though the sky was clear; and that about 3000 stones fell within a space of six by two miles.

From this time the fall of meteorites was no longer doubted. The subsequent discoveries and the present state of our knowledge are admirably stated in Fletcher's \emph{Introduction} referred to above, and can be further pursued in the special treatises on the subject.

On a review of the whole story one cannot help feeling that although the scientific proof could never have been complete without the work of Howard, and that his work was of an extraordinarily difficult nature, as is proved by its previous failure in the hands of the French chemists, yet the arguments of Chladni might have been advanced at almost any previous period had some sufficiently acute critic cared to examine the evidence without prejudice. The history traced in the foregoing pages is a curious study of the rejection of circumstantial evidence owing to its surprising nature and to the superstition with which it was mixed. The fault lay, as is clear from the official statement of the French Institute, in the refusal to accept the evidence relating to a phenomenon for which a sufficient cause could not be at once suggested --- a very common but a very dangerous attitude. Doubtless our successors will be able to regard with equal curiosity either the prejudice or the credulity with which many a problem is regarded: at the present day.

\bigskip{}

H. A. Miers.
\end{document}
